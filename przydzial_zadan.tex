Ponieważ grupa składa się tylko z dwóch osób projekt w dużej mierze będzie realizowany wspólnie. Można  jednak wyróżnić część zadań, które będą realizowane osobno. Podział prezentuje poniższa tabela (Tabela 4). Nakreśla ona wybór lidera danego zadania. Będzie on przede wszystkim odpowiedzialny za kontrolę czasu wykonania zadania.



\begin{table}[!htbp]
\begin{center}
\begin{tabular}{|c|c|c|}

\hline
\textbf{Nr} & \textbf{Nazwa zadania} & \textbf{Lider} \\ \hline\hline
Z1 & Zebranie wszystkich potrzebnych elementów do budowy robota & M. Ochman \\ \hline
Z2 & Wstępne prace nad budową podwozia robota & D.Janiak \\ \hline
Z3 & Dokończenie budowy podwozia oraz zamontowanie kół, silników wraz z enkoderami & M.Ochman \\ \hline
Z4 & Instalacja układów elektronicznych, elektrycznych oraz czujników & D.Janiak \\ \hline
Z5 & Oprogramowanie sterowania robotem & M.Ochman \\ \hline
Z6 & Oprogramowanie komunikacji Bluetooth & D.Janiak \\ \hline
Z7 & Stworzenie szkieletu aplikacji Qt z ładowaniem okna 3D & D.Janiak \\ \hline
Z8 & Ładowanie danych symulacyjnych z pliku i proste rysowanie mapy w oknie 3D & D.Janiak \\ \hline
Z9 & Stworzenie prostej aplikacji w Qt, obsługującej Bluetooth & M.Ochman \\ \hline
Z10 & Stworzenie komunikacji przez Bluetooth & D.Janiak \\ \hline
Z11 & Dostosowanie aplikacji do robota & M.Ochman \\ \hline
Z12 & System poruszania robotem & M.Ochman \\ \hline
Z13 & Sklejanie mapy 3D oraz wizualizacja ścieżki ruchu robota & D.Janiak \\ \hline
Z14 & Możliwość poruszania widokiem 3D & D.Janiak \\ \hline
Z15 & Stosowanie poprawek & M.Ochman \\ \hline

\end{tabular}
\caption{Podział zadań w grupie}
\end{center}
\end{table}

W skład grupy wchodzą osoby:
\begin{description}
\item[Marcin Ochman] - student AiR. Wybrał specjalność Robotyka, ponieważ interesuje się zarówno komputerami, elektroniką oraz nowinkami technicznymi.
	Jego głównym atutem jest umiejętność programowania w różnych językach takich jak C, C++, C\#, Python, SQL. 
	Swoje doświadczenie zdobywa zarówno na uczelni oraz pracując w firmie Vulcan.	
\item[Dorian Janiak] - Pisze od kilku lat programy w językach C++/C. Podejmował się również pracy z takimi językami jak Python, Matlab czy QML. Obecnie pracuje jako programista C++. Jego głównym zainteresowaniem jest grafika 3D (używał OpenGL i GLSL, zna podstawy RayTracingu, posługuje się programem Blender 3D). Ukończył kurs języka niemieckiego na poziomie B2.
\end{description}
