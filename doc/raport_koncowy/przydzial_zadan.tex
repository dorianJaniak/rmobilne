Ponieważ grupa składa się tylko z dwóch osób projekt w dużej mierze był realizowany wspólnie. Można  jednak wyróżnić część zadań, które miały być realizowane osobno. Podział prezentuje poniższa tabela (Tabela 4). Nakreśla ona wybór lidera danego zadania. Był on przede wszystkim odpowiedzialny za kontrolę czasu wykonania zadania.


\begin{table}[!htbp]
\begin{center}
\begin{tabular}{|c|c|c|}

\hline
\textbf{Nr} & \textbf{Nazwa zadania} & \textbf{Lider} \\ \hline\hline
Z1 & Zebranie wszystkich potrzebnych elementów do budowy robota & M. Ochman \\ \hline
Z2 & Budowa podwozia robota & M. Ochman \\ \hline
Z3 & Instalacja układów elektronicznych, elektrycznych oraz czujników & M.Ochman \\ \hline
Z4 & Oprogramowanie robota - komunikacja oraz sterowanie & M.Ochman \\ \hline
Z5 &Stworzenie programu komputerowego wizualizujący dane symulacyjne ładowane z pliku & D. Janiak \\ \hline
Z6 & Dodanie modułu, obsługującego komunikację poprzez Bluetooth & D.Janiak \\ \hline
Z7 & Dodanie możliwości sterowania robotem z poziomu programu komputerowego & D.Janiak \\ \hline
Z8 & Stosowanie poprawek & D.Janiak \\ \hline


\end{tabular}
\caption{Podział zadań w grupie}
\end{center}
\end{table}

W skład grupy wchodziły osoby:
\begin{description}
\item[Marcin Ochman] - student AiR. Wybrał specjalność Robotyka, ponieważ interesuje się zarówno komputerami, elektroniką oraz nowinkami technicznymi.
	Jego głównym atutem jest umiejętność programowania w różnych językach takich jak C, C++, C\#, Python, SQL. 
	Swoje doświadczenie zdobywa zarówno na uczelni oraz pracując w firmie Vulcan.	
\item[Dorian Janiak] - Pisze od kilku lat programy w językach C++/C. Podejmował się również pracy z takimi językami jak Python, Matlab czy QML. Obecnie pracuje jako programista C++. Jego głównym zainteresowaniem jest grafika 3D (używał OpenGL i GLSL, zna podstawy RayTracingu, posługuje się programem Blender 3D). Ukończył kurs języka niemieckiego na poziomie B2.
\end{description}
