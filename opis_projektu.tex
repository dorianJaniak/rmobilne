Projekt realizowany jest w ramach kursu Roboty mobilne. W ramach projektu założyliśmy stworzenie małego robota mobilnego wyposażonego w sonar, który umożliwi rejestrowanie mapy otoczenia (jedynie horyzontalnie).
\subsection{Konstrukcja robota}
Sonar zostanie zrealizowany przy użyciu ultradźwiękowego czujnika odległości (HC-SR04) zamontowanego na silniku krokowym. Silnik krokowy pozwala na zachowanie dużej dokładności informacji o kącie osi akustycznej. Do sterowania silnikiem krokowym posłuży sterownik ULN2003. Czujnik jest w stanie mierzyć odległość w zakresie od 2 - 200cm. Jednak nasz sonar ograniczymy do węższego zakresu, takiego w którym występuje najmniejszy błąd pomiarowy - wynik ten uzyskamy na podstawie testów. Silnik krokowy będzie zmieniał oś akustyczną czujnika, natomiast czujnik każdorazowo po wykonanym obrocie będzie wykonywał pomiary, które będą zapisywane w pamięci mikrokontrolera. Po wykonaniu pomiaru w danej pozycji robota wysyłana będzie informacja o zakończeniu pomiarów, a następnie po otrzymaniu żądania przesyłu, dane zostaną wysłane do aplikacji komputerowej poprzez moduł Bluetooth (Bluetooth HC-05). Komunikacja mikrokontrolera z modułem Bluetooth będzie odbywać się poprzez interfejs UART. W ten sposób już mikrokontroler będzie zajmował się formatowaniem danych i przygotowaniem ich według określonego przez nas protokołu komunikacyjnego. \newline 
Robot będzie typu jeżdżącego i nie będzie autonomiczny. Zastosujemy napęd na dwa koła, co pozwoli na obracanie robrota w miejscu (zostaną zamontowane w połowie długości robota). Napęd powstanie na bazie produktu Dagu RS034, w skład którego wchodzą: dwa silniki DC, dwa koła oraz enkodery. Enkodery pozwolą na zwiększenie dokładności pomiaru (odczyt będzie odbywać się w funkcjach wywoływanych przez przerwania). Sterowanie napędem odbywać się będzie przy użyciu PWM. Moduł STM32 Nucleo posiada wyjścia z funkcją PWM. Aby zachować równowagę wyposażymy konstrukcję również w kulkę podpierającą. Sterowanie robotem odbywać się będzie przez Bluetooth z poziomu aplikacji komputerowej. Sterowanie pozwoli na przynajmniej obrót robota oraz jazdę do przodu. Pozycja robota w miarę możliwości będzie weryfikowana na podstawie danych z enkoderów. Aplikacja komputerowa nie jest w tym wypadku optymalnym rozwiązaniem (lepszym byłby telefon np. z systemem Android), ponieważ zasięg Bluetooth jest stosunkowo niski, a więc użytkownik będzie zmuszony śledzić robota. Sterowanie robotem jednak nie jest głównym celem projektu.
\subsection{Cel projektu}
Projekt ma na celu umożliwienie nam zapoznania się z podstawami sterowania robotami. Projekt ma umożliwić nam nauczenie się konstruowania robotów mobilnych, zmierzenie się z problemem doboru napędu i sterowania nim, obsługi komunikacji z urządzeniem zewnętrznym i zbieranie pomiarów z czujników zewnętrznych. Wynik naszej pracy posłuży nam oraz osobom pragnącym zabrać się za tworzenie własnych robotów jako pomoc przy konstruowaniu robotów mobilnych. \newline
Założenie stworzenie sonaru jest o tyle istotne, że jest to poważny problem praktyczny. Prawie, że niezależnie od wyboru realizowanego problemu w dziedzinie robotów mobilnych robot powinien być w stanie zbierać informacje o otoczeniu. Omiatanie terenu przy pomocy jednego czujnika pozwoli mu przykładowo wybrać odpowiedni tor ruchu i ominąć przeszkody.
\subsection{Protokół komunikacyjny}
Całością procesu zarządzać będzie aplikacja zainstalowana na komputerze PC. Dane przesyłane będą poprzez Bluetooth. Komunikacja będzie się odbywać w obie strony. Aplikacja będzie żądać przede wszystkim wykonania skanowania, przesłania danych pomiarowych oraz przemieszczenia się robota. Natomiast robot będzie informować przede wszystkim, że wykonał skanowanie otoczenia oraz to w jakim szacowanym punkcie się znalazł względem początku trasy. Wszystko to zostanie opakowane w specjalny format wiadomości przedstawiony poniżej:
\newline
\textbf{W}\verb+nrZadania+\textbf{;}\verb+iloscParametrów+\textbf{;}\verb+parametr1+\textbf{;}...\textbf{;}\verb+parametrN+\textbf{;}
\newline
,gdzie W jest znakiem początkowym wiadomości, natomiast średnik znakiem oddzielającym kolejne parametry i części wiadomości.
\newline
W przypadku wiadomości, zawierającej wyniki pomiarów format jest ten sam, lecz nieparzyste parametry oznaczają kąt osi akustycznej, natomiast parzyste zmierzoną odległość, zgodnie z poniższym przykładem:\newline
\begin{verbatim}
W2;6;-30;200;-10;10;17;160;
\end{verbatim}
W powyższym przykładzie liczby -30, -10, 17 oznaczają kąt wychylenia, natomiast 200, 10, 160 zmierzoną odległość podaną w cm. 


