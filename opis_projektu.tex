Projekt realizowany jest w ramach kursu Roboty mobilne. W ramach projektu założyliśmy stworzenie małego robota mobilnego wyposażonego w sonar, który umożliwi rejestrowanie mapy otoczenia (jedynie horyzontalnie).
\subsection{Konstrukcja robota}
Sonar zostanie zrealizowany przy użyciu czujnika odległości zamontowanego na serwie. Serwo będzie zmieniało kąt czujnika, natomiast czujnik w tym czasie będzie wykonywał pomiary, które będą zapisywane w pamięci mikrokontrolera. Po wykonaniu pomiaru w danej pozycji robota dane zostaną wysłane do aplikacji komputerowej poprzez moduł Bluetooth. \newline 
Robot będzie typu jeżdżącego i nie będzie autonomiczny. Zastosujemy napęd na dwa koła, co pozwoli na obracanie robrota w miejscu. Aby zachować równowagę wyposażymy konstrukcję również w kulkę podpierającą. Sterowanie robotem odbywać się będzie przez Bluetooth z poziomu aplikacji komputerowej. Aplikacja komputerowa nie jest w tym wypadku optymalnym rozwiązaniem (lepszym byłby telefon np. z systemem Android), ponieważ zasięg Bluetooth jest stosunkowo niski, a więc użytkownik będzie zmuszony śledzić robota. Sterowanie robotem jednak nie jest głównym celem projektu.
\subsection{Cel projektu}
Cel projektu jest typowo zapoznawczy. Projekt ma umożliwić nam nauczenie się konstruowania robotów mobilnych, zmierzenie się z problemem doboru napędu i nim sterowania, obsługi komunikacji z urządzeniem zewnętrznym i zbieranie pomiarów z czujników zewnętrznych. Wynik naszej pracy posłuży nam oraz osobom pragnącym zabrać się za tworzenie własnych robotów. \newline
Założenie stworzenie sonaru jest o tyle istotne, że jest to poważny problem praktyczny. Prawie, że niezależnie od wyboru realizowanego problemu w dziedzinie robotów mobilnych robot powinien być w stanie zbierać informacje o otoczeniu. Omiatanie terenu przy pomocy jednego czujnika pozwoli mu przykładowo wybrać odpowiedni tor ruchu i ominąć duże przeszkody. 

